% Options for packages loaded elsewhere
\PassOptionsToPackage{unicode}{hyperref}
\PassOptionsToPackage{hyphens}{url}
%
\documentclass[
]{article}
\usepackage{lmodern}
\usepackage{amsmath}
\usepackage{ifxetex,ifluatex}
\ifnum 0\ifxetex 1\fi\ifluatex 1\fi=0 % if pdftex
  \usepackage[T1]{fontenc}
  \usepackage[utf8]{inputenc}
  \usepackage{textcomp} % provide euro and other symbols
  \usepackage{amssymb}
\else % if luatex or xetex
  \usepackage{unicode-math}
  \defaultfontfeatures{Scale=MatchLowercase}
  \defaultfontfeatures[\rmfamily]{Ligatures=TeX,Scale=1}
\fi
% Use upquote if available, for straight quotes in verbatim environments
\IfFileExists{upquote.sty}{\usepackage{upquote}}{}
\IfFileExists{microtype.sty}{% use microtype if available
  \usepackage[]{microtype}
  \UseMicrotypeSet[protrusion]{basicmath} % disable protrusion for tt fonts
}{}
\makeatletter
\@ifundefined{KOMAClassName}{% if non-KOMA class
  \IfFileExists{parskip.sty}{%
    \usepackage{parskip}
  }{% else
    \setlength{\parindent}{0pt}
    \setlength{\parskip}{6pt plus 2pt minus 1pt}}
}{% if KOMA class
  \KOMAoptions{parskip=half}}
\makeatother
\usepackage{xcolor}
\IfFileExists{xurl.sty}{\usepackage{xurl}}{} % add URL line breaks if available
\IfFileExists{bookmark.sty}{\usepackage{bookmark}}{\usepackage{hyperref}}
\hypersetup{
  hidelinks,
  pdfcreator={LaTeX via pandoc}}
\urlstyle{same} % disable monospaced font for URLs
\usepackage[margin=1in]{geometry}
\usepackage{graphicx}
\makeatletter
\def\maxwidth{\ifdim\Gin@nat@width>\linewidth\linewidth\else\Gin@nat@width\fi}
\def\maxheight{\ifdim\Gin@nat@height>\textheight\textheight\else\Gin@nat@height\fi}
\makeatother
% Scale images if necessary, so that they will not overflow the page
% margins by default, and it is still possible to overwrite the defaults
% using explicit options in \includegraphics[width, height, ...]{}
\setkeys{Gin}{width=\maxwidth,height=\maxheight,keepaspectratio}
% Set default figure placement to htbp
\makeatletter
\def\fps@figure{htbp}
\makeatother
\setlength{\emergencystretch}{3em} % prevent overfull lines
\providecommand{\tightlist}{%
  \setlength{\itemsep}{0pt}\setlength{\parskip}{0pt}}
\setcounter{secnumdepth}{-\maxdimen} % remove section numbering
\ifluatex
  \usepackage{selnolig}  % disable illegal ligatures
\fi

\author{}
\date{\vspace{-2.5em}}

\begin{document}

\hypertarget{important-if-the-analysis-is-completed-using-software-other-than-r-or-not-written-up-using-r-markdown-the-project-should-receive-a-0-regardless-of-its-content.}{%
\paragraph{IMPORTANT: If the analysis is completed using software other
than R, or not written up using R Markdown, the project should receive a
0 regardless of its
content.}\label{important-if-the-analysis-is-completed-using-software-other-than-r-or-not-written-up-using-r-markdown-the-project-should-receive-a-0-regardless-of-its-content.}}

\hypertarget{part-1-data-2-points}{%
\subsection{Part 1: Data (2 points)}\label{part-1-data-2-points}}

\begin{itemize}
\item
  1 pt for correct reasoning for generabizability -- Answer should
  discuss whether random sampling was used. Learners might discuss any
  reservations, those should be well justified.
\item
  1 pt for correct reasoning for causality -- Answer should discuss
  whether random assignment was used.
\end{itemize}

\hypertarget{part-2-data-manipulation-10-points}{%
\subsection{Part 2: Data manipulation (10
points)}\label{part-2-data-manipulation-10-points}}

\begin{itemize}
\tightlist
\item
  Create new variable based on \texttt{title\_type}: New variable should
  be called \texttt{feature\_film} with levels yes (movies that are
  feature films) and no (2 pt)
\item
  Create new variable based on \texttt{genre}: New variable should be
  called \texttt{drama} with levels yes (movies that are dramas) and no
  (2 pt)
\item
  Create new variable based on \texttt{mpaa\_rating}: New variable
  should be called \texttt{mpaa\_rating\_R} with levels yes (movies that
  are R rated) and no (2 pt)
\item
  Create two new variables based on \texttt{thtr\_rel\_month}:

  \begin{itemize}
  \tightlist
  \item
    New variable called \texttt{oscar\_season} with levels yes (if movie
    is released in November, October, or December) and no (2 pt)
  \item
    New variable called \texttt{summer\_season} with levels yes (if
    movie is released in May, June, July, or August) and no (2 pt)
  \end{itemize}
\end{itemize}

\hypertarget{part-3-eda-9-points}{%
\subsection{Part 3: EDA (9 points)}\label{part-3-eda-9-points}}

Conduct exploratory data analysis of the relationship between
\texttt{audience\_score} and the new variables constructed in the
previous part

\begin{itemize}
\tightlist
\item
  3 pts for plots

  \begin{itemize}
  \tightlist
  \item
    Plots should address the research questions (1 pt)
  \item
    Plots should be constructed correctly (1 pt)
  \item
    Plots should be formatted well -- size not too large, not too small,
    etc. (1 pt)
  \end{itemize}
\item
  3 pts for summary statistics

  \begin{itemize}
  \tightlist
  \item
    Summary statistics should address the research questions (1 pt)
  \item
    Summary statistics should be calculated correctly (1 pt)
  \item
    Summary statistics should be formatted well -- not taking up pages
    and pages, etc. (1 pt)
  \end{itemize}
\item
  3 pts for narrative

  \begin{itemize}
  \tightlist
  \item
    Each plot and/or R output should be accompanied by a narrative (1
    pt)
  \item
    Narrative should interpret the visuals / R output correctly (1 pts)
  \item
    Narrative should address the research question (1 pts)
  \end{itemize}
\end{itemize}

\hypertarget{part-4-modeling-15-points}{%
\subsection{Part 4: Modeling (15
points)}\label{part-4-modeling-15-points}}

Develop a Bayesian regression model to predict \texttt{audience\_score}
from the following explanatory variables. Note that some of these
variables are in the original dataset provided, and others are new
variables you constructed earlier:

\begin{itemize}
\tightlist
\item
  \texttt{feature\_film}
\item
  \texttt{drama}
\item
  \texttt{runtime}
\item
  \texttt{mpaa\_rating\_R}
\item
  \texttt{thtr\_rel\_year}
\item
  \texttt{oscar\_season}
\item
  \texttt{summer\_season}
\item
  \texttt{imdb\_rating}
\item
  \texttt{imdb\_num\_votes}
\item
  \texttt{critics\_score}
\item
  \texttt{best\_pic\_nom}
\item
  \texttt{best\_pic\_win}
\item
  \texttt{best\_actor\_win}
\item
  \texttt{best\_actress\_win}
\item
  \texttt{best\_dir\_win}
\item
  \texttt{top200\_box}
\end{itemize}

Complete Bayesian model selection and report the final model.

\begin{itemize}
\tightlist
\item
  Carrying out the model selection correctly (5 pts)
\item
  Model diagnostics (5 pts)
\item
  Interpretation of model coefficients (5 pts)
\end{itemize}

\hypertarget{prediction-5-points}{%
\subsection{Prediction (5 points)}\label{prediction-5-points}}

Pick a movie from 2016 (a new movie that is not in the sample) and do a
prediction for this movie using your the model you developed and the
\texttt{predict} function in R.

\begin{itemize}
\tightlist
\item
  Correct prediction (4 pts)
\item
  Reference(s) for where the data for this movie come from (1 pt)
\end{itemize}

\hypertarget{conclusion-3-points}{%
\subsection{Conclusion (3 points)}\label{conclusion-3-points}}

A brief summary of your findings from the previous sections
\textbf{without} repeating your statements from earlier as well as a
discussion of what you have learned about the data and your research
question. You should also discuss any shortcomings of your current study
(either due to data collection or methodology) and include ideas for
possible future research.

\begin{itemize}
\tightlist
\item
  Conclusion not repetitive of earlier statements (1 pt)
\item
  Cohesive synthesis of findings that appropriate address the research
  question stated earlier (1 pt)
\item
  Discussion of shortcomings (1 pt)
\end{itemize}

\hypertarget{overall-6-points}{%
\subsection{Overall (6 points)}\label{overall-6-points}}

\begin{itemize}
\tightlist
\item
  Uploaded HTML document resulting from the Rmd template: 1 pt
\item
  Organization: 1 pts
\item
  Readability of the text: 2 pts
\item
  Readability of the code: 2 pts
\end{itemize}

\end{document}
