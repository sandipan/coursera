\nonstopmode{}
\documentclass[a4paper]{book}
\usepackage[times,inconsolata,hyper]{Rd}
\usepackage{makeidx}
\usepackage[utf8,latin1]{inputenc}
% \usepackage{graphicx} % @USE GRAPHICX@
\makeindex{}
\begin{document}
\chapter*{}
\begin{center}
{\textbf{\huge Package `DDPQuiz3'}}
\par\bigskip{\large \today}
\end{center}
\begin{description}
\raggedright{}
\item[Type]\AsIs{Package}
\item[Title]\AsIs{What the package does (short line)}
\item[Version]\AsIs{1.0}
\item[Date]\AsIs{2014-04-23}
\item[Author]\AsIs{Who wrote it}
\item[Maintainer]\AsIs{Who to complain to }\email{yourfault@somewhere.net}\AsIs{}
\item[Description]\AsIs{More about what it does (maybe more than one line)}
\item[License]\AsIs{What license is it under?}
\end{description}
\Rdcontents{\R{} topics documented:}
\inputencoding{utf8}
\HeaderA{createmean}{This function calculates the mean}{createmean}
%
\begin{Description}\relax
This function calculates the mean
\end{Description}
%
\begin{Usage}
\begin{verbatim}
createmean(x)
\end{verbatim}
\end{Usage}
%
\begin{Arguments}
\begin{ldescription}
\item[\code{x}] is a numeric vector
\end{ldescription}
\end{Arguments}
%
\begin{Value}
the mean of x
\end{Value}
%
\begin{Examples}
\begin{ExampleCode}
x <- 1:10
createmean(x)
\end{ExampleCode}
\end{Examples}
\printindex{}
\end{document}
